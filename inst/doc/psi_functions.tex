\documentclass[a4paper,oneside,11pt,DIV=12]{scrartcl}

\usepackage{enumerate}
\usepackage[T1]{fontenc}
\usepackage{inputenc}
\usepackage{times}
\usepackage{framed}
\usepackage{amsmath,amsfonts,amssymb,bm}
\usepackage{setspace}
\usepackage[longnamesfirst]{natbib}

\newcommand{\code}[1]{{\texttt{#1}}}

\setlength\parindent{24pt}

\usepackage{hyperref}
\hypersetup{
    colorlinks=true,
    linkcolor=blue,
    filecolor=blue,
    urlcolor=blue,
    citecolor=blue
}

% ==============================================================================
\begin{document}
\shortcites{robustbase}

\title{Adding support for other $\psi$-functions}

\author{{\normalsize Tobias Schoch} \\
\begin{minipage}[t][][t]{\textwidth}
	\begin{center}
	\small{University of Applied Sciences Northwestern Switzerland FHNW} \\
	\small{School of Business, Riggenbachstrasse 16, CH-4600 Olten} \\
	\small{\texttt{tobias.schoch{@}fhnw.ch}}
	\end{center}
\end{minipage}}

\date{{\small \today, version 0.2}}
\maketitle

\renewenvironment{abstract}{%
\begin{center}\begin{minipage}{0.9\textwidth}
\rule{\textwidth}{0.4pt}
{\sffamily\bfseries\footnotesize Abstract.}\small}
{\par\noindent\rule{\textwidth}{0.4pt}\end{minipage}\end{center}}

\begin{abstract}
In this note, we explain how to add support for additional $\psi$-functions
for $M$- and $GM$-estimators of regression.
\end{abstract}

\vspace{1em}

\setstretch{1.1}

\section{Introduction}
The package (version 0.2) implements the Huber and Tukey (biweight)
$\psi$-functions. The functions are implemented in the \code{C} language,
see \code{src/psifunctions.c}. For the Huber $\psi$-function, the
``standard'' function and an asymmetric $\psi$-function are implemented.

The functions are referenced by an integer value (in the \code{C}
and \code{R} code):
\begin{itemize}
    \item \code{psi = 0}: Huber;
    \item \code{psi = 1}: asymmetric Huber;
    \item \code{psi = 2}: Tukey biweight.
\end{itemize}

\noindent For each type of $\psi$-function, the following 3 functions (in
the \code{C} language) must be defined:
\begin{itemize}
    \item \code{psi}-function (the actual $\psi$-function);
    \item \code{weight}-function associated with the $\psi$-function;
    \item \code{psi-prime}-function, the first derivative of the
        $\psi$-function.
\end{itemize}

\noindent The \code{psi}-, \code{weight}, and \code{psi-prime}-functions
have the same signature, which is shown here for a dummy function.

\begin{verbatim}
double foo(double x, const double k)
{
    # the code goes here
}
\end{verbatim}

\noindent Argument \code{x} is the function argument of $\psi(x)$, \code{k}
refers to the robustness tuning constant.

\begin{leftbar}
\noindent \textbf{Limitations:} In this note, we consider only adding support
for $\psi$-functions whose signature comply with the above dummy function.
If you want to add functions that do not comply, you have to modify the
existing code.
\end{leftbar}

\noindent The method dispatch takes place in the functions (see
\code{src/robsurvey.c}):
\begin{itemize}
    \item \code{rwlslm} ($\rightarrow$ regression coefficients and scale);
    \item \code{cov\_rwlslm} ($\rightarrow$ asymptotic covariance
        matrix)
\end{itemize}
\noindent and is implemented with function pointers.

\section{Adding another function}
In order to add support for additional $\psi$-functions (which comply with
the above signature), follow these steps:

\begin{enumerate}[1]
    \item add the \code{C} code of the \code{psi}-, \code{weight}, and
        \code{psi-prime}-functions to the package source;
    \item add an entry in the \code{switch} statement of the function
        \code{rwlslm}; it is recommended to refer the new
        function to the integers \code{psi = 3, 4, ...};
    \item add an entry in the \code{switch} statement of the function
        \code{cov\_rwlslm} referring to the same integer;
    \item add/ modify the \code{R}-functions.
\end{enumerate}

\end{document}
