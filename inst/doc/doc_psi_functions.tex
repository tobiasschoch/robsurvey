\documentclass[a4paper,oneside,11pt,DIV=12]{scrartcl}

\usepackage[utf8]{inputenc}
\usepackage{graphicx}

\setkomafont{captionlabel}{\sffamily\bfseries\small}
\setkomafont{caption}{\sffamily\small}

\usepackage[T1]{fontenc}
\usepackage{times}
\renewcommand{\familydefault}{\rmdefault}

\usepackage{amssymb,amsmath,amsthm,mathrsfs}
\usepackage{bbm}
\usepackage{bm}
\usepackage[longnamesfirst]{natbib}
\usepackage{booktabs}
\usepackage{enumerate}

% base font
\usepackage[scaled]{helvet}
\renewcommand*\familydefault{\sfdefault}

% theorems
\theoremstyle{remark}
\newtheorem*{rem}{Remark}
\newtheorem*{rems}{Remarks}

% flush floats using \afterpage{\clearpage}
\usepackage{afterpage}

% allow page breaks of long align equation environments
\allowdisplaybreaks


\usepackage{setspace}
\setlength\parindent{24pt}

\usepackage{hyperref}
\hypersetup{
    colorlinks=true,
    linkcolor=blue,
    filecolor=blue,
    urlcolor=blue,
    citecolor=blue
}

% math
\newcommand{\R}{\mathbb{R}}
\newcommand{\N}{\mathbb{N}}
\newcommand{\E}{\mathbb{E}}
\newcommand{\var}{\mathrm{var}}
\newcommand{\alert}[1]{\textbf{#1}}

% code command (can deal with '$', '_', etc.)
\makeatletter
\newcommand\code{\bgroup\@makeother\_\@makeother\~\@makeother\$\@makeother\^\@codex}
\def\@codex#1{{\normalfont\ttfamily\hyphenchar\font=-1 #1}\egroup}
\makeatother

% ==============================================================================
\begin{document}
\shortcites{anderson_bai_etal_1999}
\shortcites{blackford_petitet_etal_2002}

\title{\Large Adding Support for Other $\psi$-Functions}

\author{{\normalsize Tobias Schoch} \\
\begin{minipage}[t][][t]{\textwidth}
	\begin{center}
	\small{University of Applied Sciences Northwestern Switzerland FHNW} \\
	\small{School of Business, Riggenbachstrasse 16, CH-4600 Olten} \\
	\small{\texttt{tobias.schoch{@}fhnw.ch}}
	\end{center}
\end{minipage}}

\date{{\small \today}}
\maketitle

\renewenvironment{abstract}{%
\begin{center}\begin{minipage}{0.9\textwidth}
\rule{\textwidth}{0.4pt}
{\sffamily\bfseries\footnotesize Abstract.}\small}
{\par\noindent\rule{\textwidth}{0.4pt}\end{minipage}\end{center}}

%------------------------------------------------------------------------------
\section{Introduction}\label{ch:introduction}
\setstretch{1.1}
The \code{robsurvey} package implements the Huber and Tukey (biweight)
$\psi$-functions. The functions are implemented in the C language, see
\code{src/psifunctions.c}. For the Huber $\psi$-function, the \emph{standard}
function and an asymmetric $\psi$-function are implemented. The functions are
referenced by an integer value (in the C and R source code):

\begin{itemize}
    \item \code{psi = 0}: Huber;
    \item \code{psi = 1}: asymmetric Huber;
    \item \code{psi = 2}: Tukey biweight.
\end{itemize}

\noindent For each type of $\psi$-function, the following three functions (in
the C language) must be defined:

\begin{itemize}
    \item \code{psi}-function, $\psi(x)$, the actual $\psi$-function;
    \item \code{weight}-function, $w(x),$ associated with the $\psi$-function;
    \item \code{psi-prime}-function, the first derivative of the
        $\psi$-function, $\psi'(x)$.
\end{itemize}

\noindent The $\psi$-, $w$-, and $\psi'$-functions have the same signature,
which is shown here for a dummy function \code{foo()}.

\begin{verbatim}
double foo(double x, const double k)
{
    # the code goes here
}
\end{verbatim}

\noindent Argument \code{x} is the function argument and argument \code{k} is
the robustness tuning constant.

\begin{rem}
In this note, we consider only adding support for $\psi$-functions whose
signature comply with the above dummy function.  If you want to add
functions that do not comply, you have to modify the existing code.
\end{rem}

\noindent The method dispatch takes place in the functions (see
\code{src/psifunctions.c}):

\begin{itemize}
    \item \code{get_wgt_function()}
    \item \code{get_psi_function()}
    \item \code{get_psi_prime_function()}
\end{itemize}
\noindent and is implemented with function pointers.

\clearpage
\bibliographystyle{fhnw_en}
\bibliography{master}

\end{document}
